\section{O futuro}

%quais as próximas coisas que poderiam ser implementadas.

Se continuarmos o projeto, ou se alguém mais continuar o projeto, enxergamos alguns pontos de potencial melhoria que podem ser considerados:

\subsection{Granularidade do tempo}

Na sequência, o planejamento para esse projeto envolve melhorar a granularidade dos tempos de duração de cada tarefa. No momento, a unidade de tempo vale meia hora -- o que ainda não é realista para duração de comerciais.

Esse aumento na granularidade pode causar um crescimento inaceitável no tempo de execução das heurísticas mais ingênuas e, dessa forma, é esperado que novas necessidades apareçam -- e que, portanto, novas heurísticas ou meta-heurísticas iterativas sejam implementadas.

\subsection{Formato da entrada}

Outra mellhoria potencial seria mudar o formato da entrada para que ele contenha também informações sobre o programa, como seu nome de exibição e idade recomendada.

Mudando a entrada, será necessário melhorar as legendas do gráfico para que, com o \textit{mouse} sobre um programa, msotre-se essas novas informações, em vez no número do programa e outros caracteres, como o atual faz.

Talvez, também, dispender um tempo alterando a legenda do eixo horizontal para que ela represente, no formato em que estamos acostumados a ver, o horário em que cada programa começará e terminará.

\subsection{Novos escalonadores}

Implementar os escalonadores propostos na seção anterior, Outras Restrições, também seria uma tarefa interessante e bastante mais ligada à área de escalonamento.

Talvez até implementar alguma meta-heurística, simples a princípio, como a Busca Tabu, para compará-la com heurísticas puras e simples e decidir qual nos traria um mlehor resultado. 
