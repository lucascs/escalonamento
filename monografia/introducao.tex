\section{Introdução}

Do que vamos falar, o que é escalonamento, por que se aplica ao problema e mais um monte de besteiras...

Um escalonamento é uma forma de dividir a execução de tarefas em diferentes recursos. 

É um conceito que, no contexto de linhas de produção fabris, é usado para diminuir custos de matéria-prima ou para aumentar a velocidade ou capacidade de produção. Dadas tarefas que máquinas sabem executar, como dividi-las de forma que essas restrições sejam atendidas de maneira eficiente.

Em computação, o escalonamento de tarefas é usualmente associado à divisão do tempo de um processador ou mesmo na divisão de processos em diferentes \textit{cores} de máquinas com mais de um processador. Ainda há a extensa e vital utilização de escalonamento na paralelização de tarefas em diferentes máquinas.

Cada problema de escalonamento tem características específicas e, portanto, heurísticas e algoritmos diferentes que se aplicam melhor a cada um.

Outros trabalhos também desenvolvidos nessa matéria estudaram as diversas heurísticas, meta-heurísticas e restrições aplicáveis a problemas de escalonamento. Nesse trabalho, criamos a infraestrutura para que seja fácil implementar tais heurísticas e vê-la funcionando na prática.

De modo lúdico, com uma motivação para o problema, abordaremos as formas de escalonar programas de televisão de um canal de forma a atender às necessidades ou desejos de uma família tida como ``modelo'', apresentada ao leitor na seção de Motivação.

Em seguida, mostraremos as Heurísticas implementadas explicando a idéia do algoritmo e mostrando \textit{screenshots} dos escalonamentos gerados por essas heurísticas para uma determinada entrada do problema.

Depois, vêm as seções de Desenvolvimento, onde explicamos como a ferramenta desenvolvida é facilmente estensível para implementação de novos escalonamentos, e de Outras Restrições que poderiam ser adicionadas ao problema de modo a deixá-lo mais realista e interessante.

Finalmente, os Passos Futuros em direção aos quais o projeto pode caminhar são mencionados e a conclusão que tiramos com relação às heurísticas implementadas e ao problema abordado.
