\section{Heurísticas}

%As que foram implementadas e uma explicação sobre cada uma delas.

O problema de escalonar programas em uma grade de programação de uma emissora de TV a cabo é um caso
particular de um problema bastante conhecido de teoria da combinatória, e que também pode ser visto
como um problema de escalonamento: o \textit{Bin-packing}.

Esse problema é conhecidamente NP-difícil, então não é possível conseguir a solução ótima em um tempo
razoável. Por causa disso, ao invés de tentarmos descobrir a solução ótima, temos que ir atrás de uma
solução que é apenas satisfatória, uma aproximação. Para conseguir uma aproximação precisamos usar
heurísticas.

Heurísticas são algoritmos de aproximação, ou seja, algoritmos que não buscam encontrar a solução ótima,
mas uma solução que consiga se aproximar da ótima segundo algum fator. Esse fator é a razão entre
a solução aproximada e a ótima, significando o quão boa é a aproximação.

Uma característica importante de heurísticas é que elas tomam decisões específicas quando chegam em determinado
ponto do algoritmo. Por exemplo escolher o maior item, inverter a ordem de itens da solução, etc.

Para conseguir uma solução aproximada para o problema \textit{Bin-packing} podemos utilizar várias heurísticas.
Algumas delas serão explicadas a seguir.


